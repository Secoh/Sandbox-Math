\def\qed{{\it Q.e.d.}}

SEARCH FOR ALL 32-BIT PRIMES BY SIEVE OF ERATOSTHENES

{\it September, 2023}

Abstract

Prime numbers are widely used in cryptographic systems. However, the strength of popular chipers
depends on the selection of the number considered prime. Short primes can be guessed, and large ones
either require primality tests, which are not strict, or are available for prior study by the attacker.

One possibility to eliminate this vulnerability is to devise a system that uses many relatively short primes
in a way where the attacker must know all of them before he can read the message. 
The "short primes" must be abundant and easy to compute to support such a system.
We demonstrate in Part 1 that the regular modern computer can build and manipulate the complete list of
32-bit prime numbers. Also, Part 2 discusses the random selection of a prime within a much greater range,
specifically, 64-bit.

Let's use the sieve of Eratosthenes, the well-known and very simple method to find small prime numbers.
The very simple yet mathematically strict foundation makes the practical computer implementation a good
scholarly exercise. This article targets an audience without a mathematical background.

Part 1

A natural number is {\it prime,\/} if it divides by two numbers, $1$ and itself. ($1$ is not considered a prime.)
The series of prime numbers is infinite and starts with
$$
2, 3, 5, 7, 11, 13, 17, 19, 23, 29, 31, 37, 41,  ...
$$
In a contrast, {\it a composite\/} number can be represented by a product of two or more primes.

{\bf Observation 1.} Except for $2$ and $3$, all primes can be expressed by the following formula:
$$
6n\pm 1, \quad\hbox{where\ } n\in N \eqno (1)
$$

(Notation $\hbox{\it something}\in N$ is used to tell that is it a natural number or a positive integer number.)

{\bf Proof.} Any number can be expressed as $6n+k$, where $n$ and $k$ are natural, and $0\le k<6$. Except
for $k=1$ and $k=5$, expression $6n+k$ divides by $2$, or $3$, or both. $5$ is the same as $-1$ modulo $6$.
The prime number $5$ corresponds to $n=1$ and ``$-$'' sign in (1).
\qed

For practical applications, it is convenient to tabulate the first 9 primes, and then use (1)
starting with $n=5$. We also don't have to verify the division by $2$ and $3$ when testing the primality of the
numbers calculated by formula (1).

The sieve of Eratosthenes method tests a prime candidate by dividing it by all possible prime numbers
smaller than the number under test. The main theorem of number theory (any natural number is the product of one or
multiple primes) guarantees that {\it if\/}  a number passes Eratosthenes' sieve, {\it then\/} it also cannot be
divided by any natural number except itself or $1$.

Let $T$ be a prime candidate and $p_k$ be a known prime. By Euclid's division lemma,
$$
T=q p_k + r, \quad \hbox{where $q, r \in N,$ and\ } 0\le r<p_k. \eqno (2)
$$

If such a $p_k < T$ is found that $r=0$, the Eratosthenes' sieve stops and $T$ is {\bf not} a prime.
Let's prove, however, that the amount of testing can be further reduced.

{\bf Lemma 1.} To verify whether $T$ is a prime, it is enough to try primes up to $\sqrt{T}$.
Specifically, {\it the stop condition\/} is $p_k^2>T$. If we see the $k$-th prime that satisfies the Lemma's condition,
we don't have to test it. The last prime number we have to verify is $p_{k-1}$.

{\bf Proof.\/} Before $p_k$, we already tested all primes $2, 3, ... p_{k-1}$, and $T$ cannot be divided
by any of them. Let's show that $T$ cannot be divided by $p_k$ or by any larger number, excluding $T$ itself.

If $p>T$, the statement is trivial.

Consider an integer $p$ in the range $p_k\le p<T$. If $T$ is divisible by $p$, then the $q$ exists such that
$T=pq$. Further, $q$ can be represented as product of one or more primes: $q=q_1 q_2$, where $q_1$ is prime
and $q_2\ge 1$. That also makes $q_1\le q$ (otherwise, the equilibrium above cannot be true).
Taking all of the above together, write:
$$
p_k^2 > T = pq \ge p_k q \ge p_k q_1,
\quad \Rightarrow \quad
q_1 < p_k,
$$
and $q_1$ divides $T$, and it is prime. Contradiction. So $T$ cannot be divided by such $p$ numbers, starting with
$p_k$. \qed

{\bf Teorem 1a} The stop condition is coming from test division of $T$ by $p_k$.
Let neither of prime numbers $p_1$, $p_2$, ... $p_{k-1}$ be a divisor of $T$
For the next prime number $p_k$, let $q$ be the whole part of $T/p_k$ according to formula (2).
If the $q<p_k$, then none of $p \ge p_k$, not equal to $T$, can be a divisor of $T$.

{\bf Proof.\/} 


\bye
